\documentclass[border=2px]{standalone}
        
\usepackage[braket, qm]{qcircuit}
<<<<<<< HEAD
\usepackage{graphicx}
||||||| merged common ancestors
\usepackage{amsmath}
\pdfmapfile{+sansmathaccent.map}
% \usepackage[landscape]{geometry}
% Comment out the above line if using the beamer documentclass.
\begin{document}

\begin{equation*}
    \Qcircuit @C=1.0em @R=0.2em @!R {
	 	\lstick{ {q}_{0} :  } & \gate{\mathrm{H}} & \qw & \qw\\
	 	\lstick{ {q}_{1} :  } & \gate{\mathrm{H}} & \qw & \qw\\
	 	\lstick{c:} & \lstick{/_{_{2}}} \cw & \cw & \cw\\
	 }
\end{equation*}
=======
\usepackage{amsmath}
\pdfmapfile{+sansmathaccent.map}
% \usepackage[landscape]{geometry}
% Comment out the above line if using the beamer documentclass.
\begin{document}
% Delete the command below if there are no CP, CU1, RZZ in the circuit.
\newlength{\glen}

\begin{equation*}
    \Qcircuit @C=1.0em @R=0.2em @!R {
	 	\lstick{ {q}_{0} :  } & \gate{\mathrm{H}} & \qw & \qw\\
	 	\lstick{ {q}_{1} :  } & \gate{\mathrm{H}} & \qw & \qw\\
	 	\lstick{c:} & \lstick{/_{_{2}}} \cw & \cw & \cw\\
	 }
\end{equation*}
>>>>>>> 795491a4b721d0398dded88c5002a8781801b0bb

\begin{document} 
\scalebox{1.0}{
\Qcircuit @C=1.0em @R=0.2em @!R { \\
	 	\nghost{ {q}_{0} :  } & \lstick{ {q}_{0} :  } & \gate{\mathrm{H}} & \qw & \qw\\ 
	 	\nghost{ {q}_{1} :  } & \lstick{ {q}_{1} :  } & \gate{\mathrm{H}} & \qw & \qw\\ 
	 	\nghost{c:} & \lstick{c:} & \lstick{/_{_{2}}} \cw & \cw & \cw\\ 
\\ }}
\end{document}