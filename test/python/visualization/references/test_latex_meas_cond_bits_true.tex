\documentclass[border=2px]{standalone}

\usepackage[braket, qm]{qcircuit}
\usepackage{graphicx}

\begin{document}
\scalebox{1.0}{
\Qcircuit @C=1.0em @R=0.2em @!R { \\
	 	\nghost{{0} :  } & \lstick{{0} :  } & \qw & \gate{\mathrm{X}} & \meter & \qw & \qw\\
	 	\nghost{{1} :  } & \lstick{{1} :  } & \qw & \qw & \qw & \qw & \qw\\
	 	\nghost{\mathrm{{0} :  }} & \lstick{\mathrm{{0} :  }} & \cw & \cw & \cw & \cw & \cw\\
	 	\nghost{\mathrm{{1} :  }} & \lstick{\mathrm{{1} :  }} & \cw & \cw & \dstick{_{_{\hspace{0.0em}}}} \cw \ar @{<=} [-3,0] & \cw & \cw\\
	 	\nghost{\mathrm{{cr} :  }} & \lstick{\mathrm{{cr} :  }} & \lstick{/_{_{2}}} \cw & \cw & \cw & \cw & \cw\\
	 	\nghost{\mathrm{{4} :  }} & \lstick{\mathrm{{4} :  }} & \cw & \cw & \cw & \cw & \cw\\
	 	\nghost{\mathrm{{cs} :  }} & \lstick{\mathrm{{cs} :  }} & \lstick{/_{_{3}}} \cw & \controlo \cw^(0.0){^{\mathtt{cs_1=0x0}}} \cwx[-6] & \cw & \cw & \cw\\
\\ }}
\end{document}