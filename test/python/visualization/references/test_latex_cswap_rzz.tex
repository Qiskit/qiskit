\documentclass[border=2px]{standalone}
        
\usepackage[braket, qm]{qcircuit}

\begin{document} 

\Qcircuit @C=1.0em @R=0.2em @!R { \\
	 	\nghost{Register: q_0} & \lstick{ {q}_{0} :  } & \gate{\mathrm{X}} & \ctrl{1} & \control \qw & \qw & \qw & \qw & \qw & \qw\\ 
	 	\nghost{Register: q_1} & \lstick{ {q}_{1} :  } & \gate{\mathrm{X}} & \qswap & \ctrl{-1} & \qw & \qw & \qw & \qw & \qw\\ 
	 	\nghost{Register: q_2} & \lstick{ {q}_{2} :  } & \qw & \qswap \qwx[-1] & \ctrlo{-1} & \qw & \qw & \qw & \qw & \qw\\ 
	 	\nghost{Register: q_3} & \lstick{ {q}_{3} :  } & \qw & \qw & \ctrl{-3} & \dstick{\hspace{2.0em}\mathrm{ZZ}\,\mathrm{(}\mathrm{\frac{3\pi}{4}}\mathrm{)}} \qw & \qw & \qw & \qw & \qw\\ 
	 	\nghost{Register: q_4} & \lstick{ {q}_{4} :  } & \qw & \qw & \ctrlo{-1} & \qw & \qw & \qw & \qw & \qw\\ 
\\ }
\end{document}