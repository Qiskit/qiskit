<<<<<<< HEAD
\documentclass[border=2px]{standalone}
        
||||||| merged common ancestors
% \documentclass[preview]{standalone}
% If the image is too large to fit on this documentclass use
\documentclass[draft]{beamer}
% img_width = 5, img_depth = 8
\usepackage[size=custom,height=10,width=16,scale=0.7]{beamerposter}
% instead and customize the height and width (in cm) to fit.
% Large images may run out of memory quickly.
% To fix this use the LuaLaTeX compiler, which dynamically
% allocates memory.
=======
% \documentclass[preview]{standalone}
% If the image is too large to fit on this documentclass use
\documentclass[draft]{beamer}
% img_width = 5, img_depth = 9
\usepackage[size=custom,height=10,width=16,scale=0.7]{beamerposter}
% instead and customize the height and width (in cm) to fit.
% Large images may run out of memory quickly.
% To fix this use the LuaLaTeX compiler, which dynamically
% allocates memory.
>>>>>>> 795491a4b721d0398dded88c5002a8781801b0bb
\usepackage[braket, qm]{qcircuit}
<<<<<<< HEAD
\usepackage{graphicx}
||||||| merged common ancestors
\usepackage{amsmath}
\pdfmapfile{+sansmathaccent.map}
% \usepackage[landscape]{geometry}
% Comment out the above line if using the beamer documentclass.
\begin{document}

\begin{equation*}
    \Qcircuit @C=1.0em @R=0.2em @!R {
	 	\lstick{ {q}_{0} :  } & \gate{\mathrm{X}} & \ctrl{1} & \control \qw & \qw & \qw & \qw & \qw & \qw\\
	 	\lstick{ {q}_{1} :  } & \gate{\mathrm{X}} & \qswap & \ctrl{-1} & \qw & \qw & \qw & \qw & \qw\\
	 	\lstick{ {q}_{2} :  } & \qw & \qswap \qwx[-1] & \ctrlo{-1} & \qw & \qw & \qw & \qw & \qw\\
	 	\lstick{ {q}_{3} :  } & \qw & \qw & \ctrl{-3} & \dstick{\hspace{2.0em}\mathrm{ZZ}\,\mathrm{(}\mathrm{\frac{3\pi}{4}}\mathrm{)}} \qw & \qw & \qw & \qw & \qw\\
	 	\lstick{ {q}_{4} :  } & \qw & \qw & \ctrlo{-1} & \qw & \qw & \qw & \qw & \qw\\
	 }
\end{equation*}
=======
\usepackage{amsmath}
\pdfmapfile{+sansmathaccent.map}
% \usepackage[landscape]{geometry}
% Comment out the above line if using the beamer documentclass.
\begin{document}
% Delete the command below if there are no CP, CU1, RZZ in the circuit.
\newlength{\glen}

\begin{equation*}
    \Qcircuit @C=1.0em @R=0.2em @!R {
	 	\lstick{ {q}_{0} :  } & \gate{\mathrm{X}} & \ctrl{1} & \control \qw & \qw & \qw & \qw & \qw & \qw & \qw\\
	 	\lstick{ {q}_{1} :  } & \gate{\mathrm{X}} & \qswap & \ctrl{-1} & \qw & \qw & \qw & \qw & \qw & \qw\\
	 	\lstick{ {q}_{2} :  } & \qw & \qswap \qwx[-1] & \ctrlo{-1} & \qw & \qw & \qw & \qw & \qw & \qw\\
	 	\lstick{ {q}_{3} :  } & \qw & \qw & \ctrl{-3} & \qw & \qw & \qw & \qw & \qw & \qw\\
	 	\lstick{ {q}_{4} :  } & \qw & \qw & \ctrlo{-1} \cds{-1}{\settowidth{\glen}{\ensuremath{\mathrm{ZZ}\,\mathrm{(}\mathrm{\frac{3\pi}{4}}\mathrm{)}}} \hspace{0.5em}\hspace{\glen}\ensuremath{\mathrm{ZZ}\,\mathrm{(}\mathrm{\frac{3\pi}{4}}\mathrm{)}}} & \qw & \qw & \qw & \qw & \qw & \qw\\
	 }
\end{equation*}
>>>>>>> 795491a4b721d0398dded88c5002a8781801b0bb

\begin{document} 
\scalebox{1.0}{
\Qcircuit @C=1.0em @R=0.2em @!R { \\
	 	\nghost{ {q}_{0} :  } & \lstick{ {q}_{0} :  } & \gate{\mathrm{X}} & \ctrl{1} & \control \qw & \qw & \qw & \qw & \qw & \qw\\ 
	 	\nghost{ {q}_{1} :  } & \lstick{ {q}_{1} :  } & \gate{\mathrm{X}} & \qswap & \ctrl{-1} & \qw & \qw & \qw & \qw & \qw\\ 
	 	\nghost{ {q}_{2} :  } & \lstick{ {q}_{2} :  } & \qw & \qswap \qwx[-1] & \ctrlo{-1} & \qw & \qw & \qw & \qw & \qw\\ 
	 	\nghost{ {q}_{3} :  } & \lstick{ {q}_{3} :  } & \qw & \qw & \ctrl{-3} & \dstick{\hspace{2.0em}\mathrm{ZZ}\,(\mathrm{\frac{3\pi}{4}})} \qw & \qw & \qw & \qw & \qw\\ 
	 	\nghost{ {q}_{4} :  } & \lstick{ {q}_{4} :  } & \qw & \qw & \ctrlo{-1} & \qw & \qw & \qw & \qw & \qw\\ 
\\ }}
\end{document}